Implementation is the process of converting a new or revised system 
design into an operational one. At the present time there is no system 
as Imperial Finance which work online and provide information via web.
So this is the replacement of the manual financial system. In Imperial 
Finance most of the finance related task will be performed online.

Types of Implementation:
\begin{enumerate}
\item Implementation of a computer system to replace a manual system.
\item Implementation of a new computer system to replace an existing one.
\item Implementation of a modified application to replace an existing one.
\end{enumerate}
\vskip 0.5cm
Aspects of Implementation:
\begin{enumerate}
\item Conversion
\item Post Implementation and review
\item Software maintenance
\end{enumerate}
\vskip 0.5cm
\section{Implementation of the Project }
OFC Automentation Software is the implementation of the new system to replace
manual one. Working manually is very time consuming and irritating. The project 
implementation of Imperial Finance starts with the Administrator. 
Administrator will be the super user of the application who will 
configure system information. There will be a different interface for the employees 
and from where they can manage and view the required information.

It is a web based application, so it is distributed and data centric. 
In this application, MySQL database is used to store data related to 
employees, users offered by system, clients, etc. Since database is on 
Server, so any number of users can work simultaneously and can share 
their data with each other.
\subsection{Conversion Plan}
Conversion is the process of changing from one system to another. This 
plan involves:
\begin{enumerate}
\item Creating computer-compatible files.
\item Training the operating staff.
\item Installing terminals and hardware.
\end{enumerate}
\newpage
\subsection{Conversion Processes}
\begin{enumerate}
\item File Conversion.
\item  Data Entry.
\item User Training.
\end{enumerate}
\vskip 0.5cm
\subsection{Elements of User training}
\begin{enumerate}
\item The initial training period.
\item At the time of Installation.
\item If required, during Maintenance Phase.
\end{enumerate}

\section{Post-Implementation and Software Maintenance}
Implementation review is an evaluation of a system in terms of the 
extent to which the system accomplishes stated objectives and actual 
project costs exceeds initial estimates.
\subsection{Review Plan}
An overall plan covers following aspects:
\begin{enumerate}
\item Administrative plan.
\item Personnel requirements plan.
\item Hardware plan.
\item Documentation review plan.
\end{enumerate}
\vskip 0.5cm
After the implementation of this project, the team will see the post 
implementation phase. If there will be any concerns, those will be 
solved based on the user feedback.
\subsection{Maintenance}
In order for a software system to remain useful in its environment it 
may be necessary to carry out a wide range of maintenance activities 
upon it. There are bugs to fix, enhancement to add and optimization to 
make, changes has to be done in older version to make it application 
for current use of current version to cater the need of future. 
Maintenance can be of three types:\\
\begin{enumerate}
\item {\bf{Corrective Maintenance}}: Changes necessitated by actual errors 
(defects or residual "bugs") in a system are termed corrective 
maintenance. These defects manifest themselves when the system does not
operate as it was designed or advertised to do. A defect or “bug” can 
result from design errors, logic errors and coding errors. Design errors 
occur when for example changes made to the software are incorrect, 
incomplete, wrongly communicated or the change request misunderstood. 
In the event of a system failure due to an error, actions are taken to 
restore operation of the software system. The approach here is to locate 
the original specifications in order to determine what the system was 
originally designed to do.
\item {\bf{Adaptive Maintenance}}: Any effort that is initiated as a result of 
changes in the environment in which a software system must operate is 
termed adaptive change. Adaptive change is a change driven by the need
to accommodate modifications in the environment of the software system, 
without which the system would become increasingly less useful until it 
became obsolete. The term environment in this context refers to all the 
conditions and influences which act from outside upon the system, for 
example business rules, government policies, work patterns, software 
and hardware operating platforms. A change to the whole or part of this 
environment will warrant a corresponding modification of the software.
\item {\bf{Perfective Maintenance}}: This is actually the most common type of 
maintenance encompassing enhancements both to the function and the 
efficiency of the code and includes all changes, insertions, deletions, 
modifications, extensions, and enhancements made to a system to meet 
the evolving and/or expanding needs of the user. A successful piece of 
software tends to be subjected to a succession of changes resulting in 
an increase in its requirements. This is based on the premise that as 
the software becomes useful, the users tend to experiment with new 
cases beyond the scope for which it was initially developed. Expansion 
in requirements can take the form of enhancement of existing system 
functionality or improvement in computational efficiency. Though 
efforts have been made to develop error free systems, but no system is 
perfect, room for improvement is always there. Thus proper documentation 
for the system has been done so that it will be easy to handle any 
breakdown or any other type of system maintenance activity.

\end{enumerate}
\newpage
\section{Screenshots}
\image{0.5}{images/OFC.png}{Login}
\image{0.5}{images/OFC2.png}{Home Page}
\image{0.4}{images/OFC3.png}{New Consignment}
\image{0.5}{images/OFC5.png}{Add Payment Details}
\image{0.5}{images/OFC6.png}{Add Freight Details}
\image{0.37}{images/OFC9.png}{Consignment Note}
\image{0.4}{images/OFC10.png}{Consignment Register}
\image{0.5}{images/OFC4.png}{Search Consignment}
\image{0.5}{images/OFC7.png}{Dispatch Parcel }
\image{0.5}{images/OFC8.png}{Logout}
